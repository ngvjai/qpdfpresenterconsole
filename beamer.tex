\documentclass{beamer}
\newcommand{\beamerannot}[1]{
	\pdfannot {
		/Subtype /Text
		/Name /BeamerNote
		/H Y
		/Contents (#1)
	}
}

\title{Beamer Annotations Example}
\author{Alexandre Lissy}

\begin{document}
	\frame{\titlepage}

	\begin{frame}\frametitle{Title of the frame}
		\begin{block}{Example}
		We might discuss some data, that are tricky
		\end{block}
		\beamerannot{So let's say that they don't come from us.<br />
		This should be a new line ...<br />
		And this line should also be a new one, but very very long so that the text SHOULD wrap and not the QLabel being too wide.<br />
		We should also ask ourselves how much in height we can put as those notes. Maybe it's only a couple of lines, maybe it's a very looooong text ... Who knows ? After all, nobody tested this, so we need to do it now. I think it's becoming long enough. An interesting thing to check is what happens when it's too big in height. Points not to forget:
		<ul>
			<li>It should still be readable</li>
			<li>The speaker must not rely only on those notes</li>
			<li>It's not intended to be perfect</li>
		</ul>}
	\end{frame}

	\begin{frame}\frametitle{Even more text \ldots}
		\begin{block}{Example}
		The speaker is bad and puts everything in its notes
		\end{block}
		\beamerannot{You have to remember that notes are <strong>only</strong> a way to have an eye on <strong>important</strong> or <strong>critical</strong> informations that you don't want in the slides but than you need to talk about. This might be useful in numerous case:
		<ul>
			<li>You're giving a course to student</li>
			<li>Good Slides Practices need that you don't write too many on slides</li>
			<li>You're testing your application like me &hellip;</li>
			<li>You're having memory loss</li>
			<li>There's already too many informations on your slides</li>
			<li>And so on &hellip;</li>
			<li>Those notes should not make any use of LaTeX!</li>
			<li>Only Qt's subset of HTML for QLabel is allowed</li>
			<li>This subset is exposed in the <a href="http://doc.qt.nokia.com/latest/richtext-html-subset.html">Qt Documentation</a></li>
		</ul>}
	\end{frame}

	\begin{frame}\frametitle{Pushing limits \ldots}
		\begin{block}{Example}
		The previous slide was having enough notes but still being viewable on 1680x1050 laptop \ldots
		\end{block}
		\beamerannot{You have to remember that notes are <strong>only</strong> a way to have an eye on <strong>important</strong> or <strong>critical</strong> informations that you don't want in the slides but than you need to talk about. This might be useful in numerous case:
		<ul>
			<li>You're giving a course to student</li>
			<li>Good Slides Practices need that you don't write too many on slides</li>
			<li>You're testing your application like me &hellip;</li>
			<li>You're having memory loss</li>
			<li>There's already too many informations on your slides</li>
			<li>And so on &hellip;</li>
			<li>Those notes should not make any use of LaTeX!</li>
			<li>Only Qt's subset of HTML for QLabel is allowed</li>
			<li>This subset is exposed in the <a href="http://doc.qt.nokia.com/latest/richtext-html-subset.html">Qt Documentation</a></li>
			<li>Now, we're broking limits, to see how it goes</li>
			<li>Is this text viewable ?</li>
		</ul>}
	\end{frame}
\end{document}
